\documentclass[a4paper,11pt]{article}
\usepackage[utf8]{inputenc}
\usepackage[T1]{fontenc}
\usepackage[french]{babel}
\usepackage[right=2.5cm, left=2.5cm, bottom=4cm, top=3cm]{geometry}
\usepackage{textcomp}
\usepackage{graphicx}
\usepackage{mathtools,amssymb,amsthm}
\usepackage{lmodern}
\usepackage{ulem}
\usepackage{multirow}
\usepackage{array}
\usepackage{longtable}

\title{\vspace{13em}{\huge Calcul sécurisé - Contrôle continu}}

\begin{document}
	
	\pagenumbering{gobble}\clearpage
	\maketitle\vspace{13em}
	\newpage
	\tableofcontents
	\newpage\clearpage\pagenumbering{arabic}
	
	\section{Question 1}
	Une attaque par faute contre le DES consiste à introduire une modification sur un bit du chiffré durant l'exécution du chiffrement de manière a compromettre son execution.De ce fait si on dispose d'un chiffré correct et d'un chiffré fauté par l'attaque on peut obtenir des informations sur une partie de  la clé utilisée pour le chiffrement.
	Lors d'une attaque par force brute (recherche exhaustive) sur la clé du DES, la complexité est de $2^{56}$. l'objectif de l'attaque par faute est de réduire cette complexitée.\\

	En supposant que l'attaquant est capable d'effectuer une faute sur la valeur de sortie $R_{15}$ du $15_{ème}$ tour une attaque par faute peut être décrite de la façon suivante :\\
	
	
	\begin{center}\includegraphics[scale=0.4]{DES_feistel.png}\end{center}
	
	Lors d'une utilisation normal de DES (sans attaque par faute), on obtiendrais les resultats suivant pour $L_{16}$ et $R_{16}$.\\
	\begin{itemize}
		\item $L_{16} = L_{15} \oplus F(R_{15}, K_{16})$
		\item $R_{16} = R_{15}$
	\end{itemize}
	 
	Maintenant, si on introduit une faute sur la valeur de sortie $R_{15}$ du $15_{ème}$ tour, on obtient les valeurs suivantes :\\
	\begin{itemize}
		\item $R_{15} = R_{15}^{*}$ 
		\item $L_{16} = L_{16}^{*} = L_{15} \oplus F(R_{15}^{*}, K_{16})$
		\item $	R_{16} = R_{15}^{*}$
	\end{itemize}
	
	On a donc une possibilité de retrouver $K_{16}$ en utilisant $L_{16}$  et $L_{16}^{*}$.\\
	On va donc utiliser l'operation XOR ($\oplus$) sur $L_{16}$  et $L_{16}^{*}$ de façon a obtenir :\\
	$$L_{16}  \oplus L_{16}^{*} =  L_{15} \oplus F(R_{15}, K_{16}) \oplus  L_{15} \oplus F(R_{15}^{*}, K_{16}) $$\\
	                          $$= \xout{L_{15} } \oplus F(R_{15}, K_{16}) \oplus  \xout{L_{15} } \oplus F(R_{15}^{*}, K_{16})$$\\
	On se retrouve donc avec :\\
	$$L_{16}  \oplus L_{16}^{*} = F(R_{15}, K_{16}) \oplus F(R_{15}^{*}, K_{16})$$\\
	
	Pour continuer l'attaque on va devoir étudier la fonction F plus en détail soit :\\
	\\
	\\ \\ \\ \\ 
	$$F$$
	\begin{center}\includegraphics[scale=0.4]{FonctionF.png}\end{center}
	On constate qie la fonction F prend le bloc $R_{i - 1}$ de 32 bits en entrée ainsi que la clé $K_{i}$. $R_{i - 1}$ passe ensuite par la fonction  E qui a pour but d'appliquer une expansion sur le bloc, le passant de 32 à 48 bits.\\
	Après l'expansion, l'opération XOR est appliquée entre $R_{i - 1}$ et $K_{i}$.\\
	On a donc 48 bits , qui vont être "découpés" en 8 blocs de 6 bits. Chaque bloc va ensuite passer par une S-Box. Les S-Box prennent en entrée 6 bits et en renvoient 4, ce qui permet de ramener la nombre de bit à 32 (soit la taille initiale de $R_{i - 1}$).\\
	Ce bloc de 32 bits va finalement subir une permutation, la sortie de cette permutation étant le resultat renvoyé par la fonction F.\\
	On peut donc écrire l'équation $L_{16}  \oplus L_{16}^{*} = F(R_{15}, K_{16}) \oplus F(R_{15}^{*}, K_{16})$ de la façon suivante : \\
	$$L_{16}  \oplus L_{16}^{*} = P(S(E(R_{15})\oplus K_{16})) \oplus P(S(E(R_{15}^{*})\oplus K_{16}))$$
	
	On sait également d'après le schéma d'exécution de la fonction F qu'on peut isoler individuellement le résultat de chaque S-box, ce qui permet de  transformer cette équation.\\
	$L_{16}  \oplus L_{16}^{*} = F(R_{15}, K_{16}) \oplus F(R_{15}^{*}, K_{16})$  devient :\\
	$$L_{16} \oplus L_{16}^{*} = $$\\
	$$P(S_{1}(E(R_{15}) \oplus K_{16}^{0-5})) || S_{2}(E(R_{15}) \oplus K_{16}^{6-11}))||...)$$\\
	$$\oplus$$\\
	$$P(S_{1}(E(R_{15}^{*}) \oplus K_{16}^{0-5}))||S_{2}(E(R_{15}^{*}) \oplus K_{16}^{6-11}))||...)$$\\
	On peut se permettre d'écrire l'équation ainsi car le contenu des différentes S-box est connu, on peut donc à partir des 6 bits d'entrée trouver les bits de sortie correspondants (par ailleurs on peut également retrouver à partir des 4 bits de sortie d'une S-box plusieurs bloc de 6 bits d'entrée possible).\\
	
	Maintenant on souhaite se débarrasser de P dans notre équation, pour cela il suffit d'appliquer $P^{-1}$ à 	$L_{16}  \oplus L_{16}^{*}$ soit (possible car la permutation P est connue): \\
	$$P^{-1} (L_{16}  \oplus L_{16}^{*}) = $$\\
	$$S_{1}(E(R_{15}) \oplus K_{16}^{0-5}) \oplus S_{1}(E(R_{15}^{*}) \oplus K_{16}^{0-5})
	 || 
	S_{2}(E(R_{15})\oplus K_{16}^{6-11}) \oplus S_{2}(E(R_{15}^{*}) \oplus K_{16}^{6-11})||$$
	$$...$$\\
	
	On peut maintenant mettre en pratique le fait que l'on puisse retrouver les bits d'entrée des 6 box grâce aux bit de sortie pour utiliser cette équation sur les S-box.\\
	On va diviser $P^{-1} (L_{16}  \oplus L_{16}^{*})$ en 8 blocs de 4 bits. Chaque bloc correspond donc à une sortie de S-box. 
	On aura donc 8 équations :\\
	\begin{itemize}
		\item $P^{-1} (L_{16}  \oplus L_{16}^{*})_{0-3} = S_{1}(E(R_{15}) \oplus K_{16}))_{0-3} \oplus S_{1}(E(R_{15}^{*}) \oplus K_{16})_{0-3} $
		\item $P^{-1} (L_{16}  \oplus L_{16}^{*})_{4-7} = S_{2}(E(R_{15}) \oplus K_{16}))_{4-7} \oplus S_{2}(E(R_{15}^{*}) \oplus K_{16})_{4-7} $
		\item ...
	\end{itemize}
	$K_{16}$ est la seule valeur inconnue dans ces 8 équations. Chacune de ces équations va nous permettre de retrouver 6 bits de $K_{16}$, on va donc devoir faire une recherche exhaustive des bits de $K_{16}$ pour chaque S-box afin de retrouver les 48 bits de la sous-clé (6 bits par S-box).\\
	On fera donc 8 recherches de 6 bits pour une complexité de $8 * 2^{6} $.
	
	\section{Question 2} 
	\subsection{Décrire précisément ce que vous faites pour retrouver la clé}
	On a précédemment établi 8 équations qui devrait nous permettre de retrouver $K_{16}$, chaque équation permettant de trouver 6 bits de la clé rentrant dans la S-box correspondante.
	On va donc maintenant devoir attaquer chaque S-Box pour retrouver les 6 bits de $K_{16}$. Cependant comme vu précédemment les 4 bits de sorties d'une S-Box peuvent correspondre à plusieurs entrées de 6 bits différents. On va donc procéder de la façon suivante afin de trouver les 6 bon bits de la clé pour chaque S-Box : \\
	\begin{itemize}
		\item En premier lieu on va chercher pour chaque S-Box quels sont les chiffrés faux correspondants de la façon suivante:\\ 
		\begin{center}\includegraphics[scale=0.4]{SBOX_Finder.png}\end{center}
		\item On constate ici au résultat de $L_{16} \oplus L_{16}^{*}$ que le $1^{er}$ et le $8^{ème}$ blocs de 4 bits sont différents de 0. On en déduit donc que ce chiffré peut être utilisé pour l'attaque des S-Box 1 et 8.
		\item La partie précédente nous à permis de définir 8 équations de cette forme :
		$$P^{-1} (L_{16} \oplus L_{16}^{*})_{0-3} = S_{1}(E(R_{15})\oplus K_{16})_{0-3} \oplus S_{1}(E(R_{15}^{*})\oplus K_{16})_{0-3}...$$\\
		On a vu comment identifier quels chiffrés faux utiliser contre quels S-BOX, on va à partir de la devoir trouver tout les couples de 6 bits possibles $R_{15}\oplus K_{16}$ et $R_{15}^{*}\oplus K_{16}$ tel que :
		$$S(E(R_{15})\oplus K_{16})_{i-i+3} \oplus S(E(R_{15}^{*})\oplus K_{16})_{i-i+3} = P^{-1} (L_{16} \oplus L_{16}^{*})_{i-i+3}$$
		On va ensuite devoir isoler $K_{16}$.\\
		Pour ce faire on va :\\
		\begin{itemize}
			\item Récupérer le chiffré juste C. On notera\\ 
			$L_{16} = IP(C)_{0-31}$  et $ R_{15} = R_{16}  = IP(C)_{32-62}$\\
			\item On récupère un chiffré faux FC utilisable sur la s-box a attaquer. On notera \\
			 $L_{16}^{*} = IP(FC)_{0-31}$  et $ R_{15}^{*} = R_{16}^{*} = IP(FC)_{32-62}$\\
			\item Chacune de ces variables est connue est fait 32 bits.\\
			On va donc prendre en référence $P^{-1}(L_{16} \oplus L_{16}^{*})$\\
			O va également appliquer une expansion E (présente dans la fonction F vue précédemment) à $R_{15}$ et $R_{15}^{*}$ tel que $E\_R_{15} = E(R_{15})$ et $E\_R_{15}^{*} = E(R_{15}^{*})$ avec $E\_R_{15}$ et $E\_R_{15}^{*}$ de 48 bits.\\
			\item On effectue ensuite une recherche exhaustive de $K_{16}$.\\
			On note $Ctmp = E\_R_{15} \oplus K_{16}$ \\
			et \ \ \ \ \ \ \ \ $FCtmp = E\_R_{15}^{*} \oplus K_{16}$, chacune de ces 2 variables composée de 48 bits.\\
			\item Maintenant, on va récupérer 6 bits correspondant à la S-box que l'on souhaite attaquer (les 6 premiers si on souhaite attaquer $S_{1}$ et ainsi de suite).
			\item On utilise ces 6 bits sur la S-Box à attaquer avec Ctmp et FCtmp. Puis on appliquer un XOR sur le resultat obtenue avec Ctmp et FCtmp sur la S-Box.\\
			Si le résultat du XOR correspond à  $P^{-1}(L_{16} \oplus L_{16}^{*})$ alors on saura que les 6 bits de $K_{16}$ utilisé pour créer Ctmp et FCtmp sont une solutions possible.
			\item Enfin on réitère ce processus avec plusieurs chiffrés différents sur une même S-box (les chiffrés identifiés précédemment comme étant utilisable sur cette S-Box).\\
			On aura donc plusieurs solutions de 6 bits possibles pour $K_{16}$. Les 6 bon bits a conservé seront ceux communs a chaque attaque effectuée sur la S-Box. 
		\end{itemize}
		\item De cette façon on a réussi a identifier 6 bits de $K_{16}$ grâce à une S-Box. Il nous suffit donc d'attaquer les 7 autres de la même manière pour réussir à obtenir 8 * 6 bits = 48 bits soit $K_{16}$.
	\end{itemize} 
	\subsection{Donnez les 48 bits de clé que vous obtenez grâce à cette attaque par fautes}
	On à pu, grâce à l'attaque décrite précédement identifier $K_{16}$ comme étant :\\
	\begin{itemize}
		\item directement en sortie du programme : $$b|8|3a|21|??|d|2a$$
		\item converti en binaire : $$001011|001000|111110|110110|100001|?|001101|101010$$
		\item converti en hexadécimal : $$? $$
	\end{itemize}
	
	\section{Question 3}
	\subsection{Expliquer comment on peut retrouver les 8 bits manquants}
	\subsection{Faites-le, et donner ainsi la valeur complète de la clé qui vous a été assignée.}
	
	\section{Question 4}
	
	\section{Question 5}

\end{document}
